\documentclass[times, utf8, zavrsni, numeric]{fer}

\usepackage{booktabs}
\usepackage{listings}

\usepackage{xcolor}
\usepackage{color, colortbl}
\usepackage{pgfplotstable}
\usepackage{pgfplots}

\usepackage{url}
\def\UrlBreaks{\do/\do-}
\usepackage{breakurl}
\usepackage[breaklinks]{hyperref}

\hypersetup{
	colorlinks,
	linkcolor={black},
	citecolor={black},
	urlcolor={black}
}

\definecolor{codegray}{rgb}{0.5,0.5,0.5}
\definecolor{codegreen}{rgb}{0,0.6,0}
\definecolor{codepurple}{rgb}{0.58,0,0.82}
\definecolor{lightblue}{rgb}{0.7,0.99,0.99}

\lstdefinestyle{mystyle}{
	frame=single,   
	commentstyle=\color{codegreen},
	keywordstyle=\color{magenta},
	stringstyle=\color{codepurple},
	basicstyle=\ttfamily,
	breakatwhitespace=false,         
	breaklines=true,                 
	captionpos=b,                    
	keepspaces=true,                 
	showspaces=false,                
	showstringspaces=false,
	showtabs=false,                  
	tabsize=2
}

\lstset{style=mystyle}

\renewcommand*{\lstlistingname}{Isječak koda}
\renewcommand*\lstlistlistingname{Popis isječaka koda}

\begin{document}

% TODO: Navedite broj rada.
\thesisnumber{1}

% TODO: Navedite naslov rada.
\title{Bežični prijenos audio signala putem BLE sučelja razvojnog sustava STM32WB5MM-DK}

% TODO: Navedite vaše ime i prezime.
\author{Jelena Gavran}

\maketitle

% Ispis stranice s napomenom o umetanju izvornika rada. Uklonite naredbu \izvornik ako želite izbaciti tu stranicu.
%\izvornik

% Dodavanje zahvale ili prazne stranice. Ako ne želite dodati zahvalu, naredbu ostavite radi prazne stranice.
\zahvala{}

\tableofcontents

\begingroup
\renewcommand*\listfigurename{Popis slika}
\listoffigures
\addcontentsline{toc}{chapter}{\lstlistlistingname}
\lstlistoflistings
\listoftables
\endgroup




\chapter{Uvod}


Ovaj završni rad dio je doktorskog rada koji se bavi analizom hrkanja. Za provođenje istraživanja potrebno je sakupiti bazu zvučnih zapisa nad kojima će se provoditi obrada. Budući da se istraživanje i snimanje provode u bolnici, gdje postoje ograničenja nad uređajima koji nadziru i snimaju pacijente, potrebno je razviti neinvazivno softversko i hardversko rješenje koje će snimati i slati podatke na udaljenu lokaciju. Kao uređaj za snimanje zvuka odabran je mikrokontroler STM32WB5M koji će snimati hrkanje i slati snimljeni zvučni zapis Bluetoothom na računalo.  

U okviru ovog završnog rada razvijena je programska potpora za mikrokontroler STM32WB5M te je uspostavljeno BLE komunikacijsko sučelje između razvojnog sustava i osobnog računala s operacijskim sustavom Linux. Sučeljem se prenosi zvučni signal sniman MEMS mikrofonom s mikrokontrolera na računalo. Također je i razvijeno korisničko sučelje za pokretanje komunikacije, prijem i pohranu signala. Blok shema sustava prikazana je na Slici 1.1. 

\begin{figure}[ht]
	\includegraphics[width=\linewidth]{imgs/shema}
	\caption{Blok shema sustava}
	 \label{fig:shema}
\end{figure}

\eject
\chapter{STM32WB5MM-DK razvojni sustav}

Razvojni sustav temelji se na STM32WB5MMG modulu tvrtke \textit{STMicroelectronics}, koji je dio linije STM32WBx5 razvojnih sustava. Kao i svi mikrokontroleri iz te skupine, modul sadrži 32-bitni Arm Cortex-M4, aplikacijski procesor koji radi na frekvenciji do 64 MHz, te Cortex-M0+, mrežni procesor s frekvencijom rada do 32 MHz. Modul sadrži 1 MB \textit{Flash} memorije i 256 KB SRAM-a. Budući da je modul RF primopredajnik, podržava protokole Bluetooth, Zigbee, Thread i konkurentne bežične standarde. Sustav također ima 0.96-inčni 128x64 zaslon, RGB LED lampice te senzore za temperaturu, dodir, I2C, \textit{Time-of-Flight} senzor i žiroskop. Od ostale periferije najznačajniji je digitalni MEMS mikrofon. STM32 modul je višeprotokolni, bežični uređaj niske potrošnje energije (engl. \textit{ultra-low-power}) primarno namijenjen razvoju aplikacija koje koriste audio, USB ili \textit{Bluetooth Low Energy} (BLE) protokol. 

\begin{figure}[ht]
	\includegraphics[width=\linewidth]{imgs/discovery_kit}
	\caption{Konfiguracija STM32WB5MM-DK razvojnog sustava}
	\label{fig:discovery-kit}
\end{figure}

\section{BLE protokol}

Bluetooth protokol korišten je za povezivanje razvojnog sustava s matičnim računalom i za prijenos audio signala s mikrokontrolera. BLE je vrsta bežične komunikacije namijenjena komunikaciji kratkog dometa s niskom potrošnjom energije. Razvijen je kako bi se postigao standard vrlo male snage koji radi s baterijom veličine kovanice (engl. \textit{coin-cell batteries}) nekoliko godina.
Klasična Bluetooth tehnologija razvijena je kao bežični standard, što je omogućilo razvoj bežičnih i prenosivih uređaja, no ne podržava dug život baterije zbog brze i nepredvidive komunikacije te složenih postupaka povezivanja. BLE uređaji troše samo dio energije koju troše standardni Bluetooth proizvodi te omogućavaju malenim uređajima s malim baterijama bežično povezivanje s uređajima koji koriste klasični Bluetooth.

BLE radi u istom opsegu od 2,4 GHz kao i standardni Bluetooth, no koristi različite kanale od standardnog Bluetootha. Koristi 40 kanala od 2 MHz za prijenos podataka korištenjem modulacije Gaussova pomaka frekvencije (metoda koja se koristi za glatkije prijelaze između podatkovnih impulsa), zbog čega skokovi frekvencije proizvode manje smetnji u usporedbi sa standardnom Bluetooth komunikacijom.

Arhitektura BLE tehnologije naziva se još i BLE stog zbog slojevite strukture. Stog se sastoji od dvije glavne komponente:
\begin{itemize}
	\item Upravljač (engl. \textit{Controller})
	\item Domaćin (engl. \textit{Host})
\end{itemize}

Upravljač se sastoji od fizičkog sloja i sloja veze. Host uključuje protokol kontrole i prilagodbe logičke veze (L2CAP), upravitelja sigurnosti (SM), protokol atributa (ATT), generički profil atributa (GATT) i generički profil pristupa (GAP). Sučelje između komponenti naziva se sučelje host kontrolera (HCI).


\begin{figure}[ht]
		\centering
		\includegraphics[scale=0.5]{imgs/ble_stack_arch}
		\caption{BLE arhitektura stoga}
		\label{fig:ble-stack-arch}
\end{figure}

\subsection{Upravljač}
\subsubsection{Fizički sloj}
Fizički sloj je radio brzine od 1 Mbps koji prenosi informacije GFSK (\textit{Gaussian Frequency Shift Keying}) frekvencijskom modulacijom. Radi u 2,4 GHz ISM pojasu bez licence na 2400-2483,5 MHz. 
BLE sustav koristi 40 RF kanala (0-39), s razmakom od 2 MHz. Postoje dvije vrste kanala:
\begin{enumerate}
	\item Kanali za oglašavanje koji koriste tri fiksna RF kanala (37, 38 i 39) za
	\begin{enumerate}
		\item Pakete kanala za oglašavanje
		\item Pakete korištene za otkrivanje ili povezivanje
		\item Pakete korištene za odašiljanje ili skeniranje
	\end{enumerate}
	\item Podatkovni fizički kanal, koristi ostalih 37 RF kanala za dvosmjernu komunikaciju između povezanih uređaja.
\end{enumerate}

BLE je tehnologija adaptivnog skakanja frekvencije (AFH) koja može koristiti samo podskup svih dostupnih frekvencija kako bi se izbjegle sve frekvencije koje koriste druge neprilagodljive tehnologije. To omogućuje prelazak s lošeg kanala na poznati dobar kanal korištenjem specifičnog algoritma za skakanje frekvencije, koji određuje sljedeći dobar kanal za korištenje.

\subsubsection{Sloj veze}
Sloj veze određuje kako dva uređaja mogu koristiti radio za međusoban prijenos informacija. Također definira automat s pet stanja:
\begin{itemize}
	\item Stanje pripravnosti (engl. \textit{Standby}): uređaj ne šalje niti prima pakete
	\item Oglašavanje: uređaj šalje oglase putem kanala za oglašavanje
	\item Skeniranje: uređaj traži uređaje oglašivača
	\item Pokretanje (iniciranje): uređaj pokreće vezu s uređajem oglašivača
	\item Veza: uređaj inicijatora u glavnoj je ulozi (engl. \textit{master}) - komunicira s uređajem u podređenoj (engl. \textit{slave}) ulozi i definira vrijeme prijenosa
	\item Uređaj oglašivača je u ulozi podređenog - komunicira s jednim uređajem u glavnoj ulozi 
\end{itemize}

\begin{figure}[ht]
	\centering
	\includegraphics[]{imgs/ll_state_machine}
	\caption{Automat sloja veze}
	\label{fig:ll-state-machine}
\end{figure}

\subsubsection{HCI}
Sloj sučelja glavnog kontrolera (HCI) pruža sredstvo komunikacije između hosta i upravljača  putem softverskog API-ja ili hardverskog sučelja kao što su: SPI, UART ili USB. Dolazi iz standardnih Bluetooth specifikacija, s novim dodatnim naredbama za funkcije specifične uz nisku potrošnju energije.

\subsection{Domaćin}

\subsubsection{L2CAP}
Protokol logičke veze i sloja prilagodbe (L2CAP) podržava multipleksiranje protokola više razine, operacije fragmentacije paketa i ponovnog sastavljanja, te prijenos informacija o kvaliteti usluga.

\subsubsection{ATT}
Protokol atributa (ATT) omogućuje uređaju da prikazuje podatke, koji se nazivaju atributima, drugom uređaju. Uređaj koji prikazuje atribute naziva se poslužiteljem, a uređaj koji ih koristi naziva se klijentom. 
ATT definira skup metoda za otkrivanje, čitanje i pisanje atributa na drugi uređaj. Implementira \textit{peer-to-peer} protokol između poslužitelja i klijenta tipičnom zahtjev-odgovor strukturom. Poslužitelj i klijent imaju sljedeće uloge: 
\begin{itemize}
	\item Poslužitelj
	\begin{itemize}
		\item Sadrži sve atribute (baza atributa)
		\item Prima zahtjeve, šalje odgovore, izvršava naredbe
		\item Obavijesti o promjeni vrijednosti atributa
	\end{itemize}
	\item Klijent
	\begin{itemize}
		\item Komunicira s poslužiteljem
		\item Šalje zahtjeve, čeka na odgovor
		\item Može čitati i mijenjati podatke uz dopuštenje poslužitelja
	\end{itemize}
\end{itemize}

\subsubsection{SM}
BLE sloj veze podržava enkripciju i autentifikaciju korištenjem načina brojača s CBC-MAC algoritmom (kod za provjeru autentičnosti lančanih poruka) i 128-bitnu AES blok šifru (AES-CCM). Kada se enkripcija i autentifikacija koriste u vezi, 4-bajtna provjera integriteta poruke (MIC) dodaje se korisnom učitavanju podatkovnog kanala PDU. Enkripcija se primjenjuje i na polja od PDU i MIC. Kada dva uređaja žele šifrirati komunikacije tijekom veze, upravitelj sigurnosti (SM) koristi postupak uparivanja. Ovaj postupak omogućuje provjeru autentičnosti dvaju uređaja razmjenom informacija o njihovu identitetu kako bi se stvorili sigurnosni ključevi koji se mogu koristiti kao osnova za pouzdani odnos ili (jednu) sigurnu vezu. 

\subsubsection{GATT}
Generički atributni profil (GATT) definira okvir za korištenje ATT protokola, a koristi se za usluge, otkrivanje deskriptora, čitanje, pisanje i obavijesti. U GATT kontekstu, kada su dva uređaja povezana, postoje dvije uloge uređaja:
\begin{itemize}
	\item GATT klijent: uređaj pristupa podacima na udaljenom GATT poslužitelju putem čitanja, pisanja, obavještavanja 
	\item  GATT poslužitelj: uređaj pohranjuje podatke lokalno i pruža metode pristupa podacima udaljenom GATT klijentu
\end{itemize}

Uređaj istovremeno može biti i GATT poslužitelj i GATT klijent.

\subsubsection{GAP}
Bluetooth sustav definira osnovni profil koji implementiraju svi Bluetooth uređaji koji se naziva generički profil pristupa (GAP), koji definira osnovne zahtjeve Bluetooth uređaja. Postoje četiri uloge GAP profila:
\begin{itemize}
	\item Emiter: šalje oglase
	\item Promatrač: prima oglase
	\item Periferija: uvijek u načinu oglašavanja i u \textit{slave} ulozi 
	\item Centar: nikada ne šalje oglase, uvijek u \textit{master} ulozi
\end{itemize}

U kontekstu GAP-a definirana su dva temeljna koncepta:
\begin{itemize}
	\item GAP načini rada (engl. \textit{modes}): konfigurira uređaj da djeluje na određeni način duži vremenski period. Postoje četiri tipa GAP načina rada: emitiranje, otkrivanje, povezivanje i vezanje
	\item GAP procedure: konfigurira uređaj da izvrši jednu radnju u ograničenom vremenskom periodu. Postoje četiri tipa GAP postupaka: promatrač, otkrivanje, povezivanje, postupci povezivanja
\end{itemize}

Istovremeno se mogu koristiti različiti načini otkrivanja i povezivanja.

\section{MEMS mikrofon}

MEMS (\textit{Micro-Electro-Mechanical Systems}) mikrofon je elektroakustični pretvornik koji sadrži MEMS senzor i aplikacijski specifičan integrirani sklop (ASIC). MEMS mikrofoni se uglavnom temelje na elektretskim kapsulama i obično imaju ugrađena pretpojačala i analogno-digitalne pretvornike. MEMS mikrofoni su također poznati kao mikrofonski čipovi ili silikonski mikrofoni.

Svi mikrofoni detektiraju akustične valove pomoću fleksibilne membrane, odnosno dijafragme. Membrana se pomiče pod pritiskom induciranih akustičnih valova. Danas većina MEMS mikrofona na tržištu koristi kapacitivnu tehnologiju za mjerenje zvuka. Kapacitivni MEMS mikrofoni mjere kapacitet između fleksibilne mikromembrane i fiksne stražnje ploče. Promjene tlaka zraka koje stvaraju zvučni valovi uzrokuju pomicanje membrane. Stražnja ploča je perforirana kako bi kroz nju mogao strujati zrak i dizajnirana je da ostane kruta budući da zrak prolazi kroz njezine perforacije. Kako se membrana pomiče, kapacitet se mijenja između pokretne membrane i fiksne stražnje ploče (budući da se udaljenost između njih mijenja), a ta se promjena može analizirati i zabilježiti.

Dizajn digitalnog MEMS mikrofona obično ima dodatni CMOS čip kao analogno-digitalni pretvornik. Ovi čipovi učinkovito preuzimaju pojačane analogne audio signale i pretvaraju ih u digitalne podatke. Također omogućuju lakšu integraciju digitalnih MEMS mikrofona s digitalnim proizvodima.

Najčešći format za digitalno kodiranje unutar MEMS mikrofona je modulacija trajanja impulsa (PDM). PDM omogućuje komunikaciju jednom podatkovnom linijom i satom. Prijemnici PDM signala, kao i sami MEMS mikrofoni, jeftini su i lako dostupni.

\subsection{MEMS tehnologija}
Mikroelektromehanički sustavi ili MEMS je tehnologija koja se definira kao sustav minijaturiziranih mehaničkih i elektromehaničkih elemenata (tj. uređaja i struktura) koji su izrađeni mikrotvorničkim tehnikama. Fizičke dimenzije MEMS uređaja mogu varirati od znatno ispod jednog mikrometra pa sve do nekoliko milimetara. Isto tako, tipovi MEMS uređaja mogu varirati od relativno jednostavnih struktura bez pokretnih elemenata, do iznimno složenih elektromehaničkih sustava s više pokretnih elemenata pod kontrolom integrirane mikroelektronike. Jedan glavni kriterij MEMS-a je da postoje barem neki elementi koji imaju neku vrstu mehaničke funkcionalnosti bez obzira na to mogu li se ti elementi kretati ili ne.

Dok su funkcionalni elementi MEMS-a minijaturizirane strukture, senzori, aktuatori i mikroelektronika, najznačajniji elementi su mikrosenzori i mikroaktuatori. Oni su  kategorizirani kao pretvornici energije iz jednog oblika u drugi - primjerice mikrosenzor, koji obično pretvara izmjereni mehanički signal u električni.

Stvarni potencijal MEMS-a ostvaruje se kada se minijaturizirani senzori, aktuatori i strukture spoje na silicijsku podlogu zajedno s integriranim krugovima, odnosno mikroelektronikom. Dok se elektronika proizvodi pomoću sekvenci procesa integriranog kruga (npr. CMOS, bipolarni ili BICMOS procesi), mikromehaničke komponente proizvode se korištenjem kompatibilnih \textit{micromachining} procesa koji selektivno urezuju dijelove silikonske pločice ili dodaju nove strukturne slojeve kako bi formirali mehaničke i elektromehaničke uređaje. Još je kompleksnije ako se MEMS može spojiti ne samo s mikroelektronikom, već i s drugim tehnologijama kao što su fotonika, nanotehnologija itd. To se ponekad naziva heterogenom integracijom. Dok su složenije razine integracije budući trend MEMS tehnologije, sadašnja je tehnologija skromnija i obično uključuje jedan diskretni mikrosenzor, jedan diskretni mikroaktuator, jedan mikrosenzor integriran s elektronikom, mnoštvo identičnih mikrosenzora integriranih s elektronikom, jedan mikroaktuator integriran s elektronikom ili mnoštvo identičnih mikroaktuatora integriranih s elektronikom. 
 

\subsection{Rad MEMS mikrofona}

MEMS mikrofon sadrži sljedeće komponente:
\begin{itemize}
	\item MEMS pretvornik: sastoji se od membrane, perforirane ploče i ležišta
	\item Isprintana matična ploča (PCB): uključuje ASIC polarizacijsku jedinicu, mikrofonsko pretpojačalo i AD pretvornik
	\item Mehanički poklopac
\end{itemize}

Zvučni valovi ulaze u MEMS mikrofon kroz poklopac i prolaze kroz perforirano kućište i ploču prije nego dođu do membrane. Valovi uzrokuju različiti zvučni tlak na membrani i razliku u tlaku između prednje i stražnje strane membrane. Ova razlika tlaka uzrokuje njeno pomicanje sukladno zvučnim valovima. Međutim, mikrofonski signal se stvara samo ako postoji naboj između vodljive membrane i nepokretne ploče. Ploča i membrana skupa djeluju kao kondenzator koji je potrebno napuniti za ispravan rad. ASIC osigurava ovo punjenje.

Jednom napunjene, ploča i membrana mogu proizvesti napon. Budući da djeluju kao kondenzator, svaka promjena kapaciteta prouzročit će obrnuto proporcionalnu promjenu napona. Kapacitet je funkcija udaljenosti između ploče i membrane, stoga dok membrana oscilira, stvara se izmjenični napon odnosno mikrofonski signal. Ovaj napon treba pojačati da bi bio koristan kao audio signal, stoga odvojeni integrirani krug (poluvodička matrica), uključen u PCB, pojačava signal.

U analognom MEMS mikrofonu, pojačani signal bi se zatim izveo iz MEMS mikrofona i poslao kamo treba ići. Međutim, u digitalnom MEMS mikrofonu postoji dodatni proces u kojem ADC pretvara analogni signal (putem PDM) prije nego što emitira digitalni audio signal.

Kao što se vidi na Slici 2.4., stacionarna ploča je perforirana, što omogućava zraku prolaz do membrane. Na ovom je prikazu ASIC čip pričvršćen na ploču, no to nije slučaj kod svakog MEMS mikrofona. 

Stražnja je komora u ovom primjeru zatvorena, što znači da je MEMS mikrofon tlačni mikrofon - membrana je otvorena samo za zvučne valove s jedne strane, što znači da prima zvuk iz svih smjerova. Stražnja komora također djeluje kao akustični rezonator i tako pomaže pri pravilnom podešavanju mikrofona.

Također je potreban i otvor za ventilaciju kako bi stražnja komora bila pod tlakom okoline.

\begin{figure}[ht]
	\includegraphics[width=\linewidth]{imgs/mems_mic}
	\caption{Poprečni presjek MEMS mikrofona}
	\label{fig:mems-mic}
\end{figure}
\chapter{Povezivanje razvojnog sustava i računala}

Računalo i STM32 razvojni sustav dva su odvojena sustava koja moraju međusobno komunicirati i razmjenjivati podatke. Za ostvarenje njihove veze razvijena su dva programska rješenja:
\begin{enumerate}
	\item programska potpora za mikrokontroler, koja će omogućiti pokretanje i snimanje zvučnog zapisa te njegov prijenos BLE sučeljem,
	\item programska potpora za računalo, koja će ostvariti Bluetooth vezu između računala i mikrokontrolera te omogućiti prijem i pohranu primljenog audio signala.
\end{enumerate} 

\section{Programska potpora za mikrokontroler}

Dvije glavne funkcionalnosti koje mikrokontroler mora sadržavati su snimanje zvuka i njegov prijenos BLE komunikacijskim sučeljem. Za rad mikrokontrolera odabran je paket funkcija \textit{FP-AUD-BVLINKWB1} iz alata \textit{STM32Cube} koji je razvila tvrtka \textit{STMicroeletronics}. Ovaj \textit{firmware} omogućava potpuni dvosmjerni prijenos zvuka koji se prenosi BLE sučeljem koristeći Opus algoritam za kompresiju. Aplikacija sadrži \textit{drivere} i posrednički softver (engl. \textit{middleware}) za BLE i digitalne MEMS mikrofone. Također uključuje kompletan Opus audio kodek kao \textit{middleware} za izvođenje dvosmjernog i simultanog prijenosa zvuka između dva STM32WB mikrokontrolera. 

\subsection{Arhitektura programske potpore za mikrokontroler}

Softver se temelji na sloju apstrakcije hardvera STM32CubeHAL za STM32 mikrokontroler. Paket funkcija opremljen je skupom \textit{middleware} komponenti za audio prijem, kompresiju i
dekompresiju, prijenos podataka preko BLE sučelja i USB-a.

Aplikacija se sastoji od sljedećih slojeva softvera:
\begin{itemize}
	\item STM32Cube HAL sloj: pruža jednostavan i generički skup generičkih i proširenih API-ja (sučelja za programiranje aplikacije) za interakciju s gornjim slojevima aplikacije i bibliotekama. Ovi su API-ji izgrađeni na zajedničkoj arhitekturi te je moguće na njih dodavati slojeve (primjerice specifični \textit{middleware}) bez potrebe za specifičnim hardverskim informacijama mikrokontrolera.
	\item Sloj paketa podrške za ploču (BSP): skup API-ja koji pruža programsko sučelje za periferne uređaje specifične za ploču kao što su SPI, ADC, LED i korisnički gumbi.
\end{itemize}

\begin{figure}[ht]
	\includegraphics[width=\linewidth]{imgs/firmware_software_arch}
	\caption{Arhitektura softvera FP-AUD-BVLINKWB1}
	\label{fig:firmware_software_arch}
\end{figure}

Komponente za obradu funkcijskog paketa \textit{FP-AUD-BVLINKWB1} dizajnirane su za stvaranje bežične audio veze između modula odašiljača (Tx) i prijemnika (Rx), gdje mikrokontroler služi kao odašiljač, a računalo kao prijemnik. Cijeli lanac obrade zvuka počinje prijemom MEMS digitalnim mikrofonom i kulminira reprodukcijom zvuka na računalu.

BLE je konfiguriran za slanje paketa s maksimalnom veličinom od 150 bajtova. Ovisno o aplikaciji, kodirani bajtovi mogu biti iznad ovog praga, stoga komprimirani međuspremnik (engl. \textit{buffer}) mora biti podijeljen u više BLE paketa. Štoviše, veličina kodiranog međuspremnika može promijeniti svaki audio okvir i prijemnik mora znati njegovu duljinu da bi ga obnovio; za ovaj opseg implementiran je jednostavan protokol BLE prijenosa.

Na strani odašiljača, zvuk se dobiva digitalnim MEMS mikrofonom kao 1-bitni PDM signal i pretvara se pomoću filtra za pretvorbu PDM-u-PCM u 16-bitni PCM. Svaki put kad je audio okvir spreman, prenosi se u algoritam kompresije: veličina kodiranog međuspremnika koju vraća Opus koder može se značajno promijeniti u skladu s parametrima Opus kodera.

\begin{figure}[ht]
	\includegraphics[width=\linewidth]{imgs/duplex_chain}
	\caption{Lanac obrade odašiljača u \textit{FP-AUD-BVLINKWB}}
	\label{fig:duplex_chain}
\end{figure}

\subsection{\textit{Middleware} za prijenos zvuka}

Budući da \textit{streaming} zvuka nije dio predefiniranog skupa profila mikrokontrolera, \textit{FP-AUD-BVLINKWB1} definira uslugu specifičnu za dobavljača pod nazivom \textit{BlueVoiceOPUS} koja je posrednik između korisničkog zvuka i  klijentskog uređaja. Ovisno o pokrenutoj aplikaciji, karakteristika se mijenja između audio ili glazbene karakteristike. Budući da \textit{streaming} glazbe nije implementiran u ovoj aplikaciji, opisane su samo audio karakteristike. 

Audio karakteristike sadrže sljedeće atribute:
\begin{itemize}
	\item \textit{Att1} sadrži deklaraciju audio karakteristika
	\begin{itemize}
		\item UUID: standardni 16-bitni UUID za karakterističnu deklaraciju
		\item Dozvole: R
		\item Vrijednost: svojstva za ovu karakteristiku su "\textit{notify only}", a UUID je za audio podatke
	\end{itemize}
	\item \textit{Att2} sadrži audio podatke
	\begin{itemize}
		\item UUID: isti UUID u zadnjih 16 bajtova vrijednosti atributa definicije karakteristike
		\item Dozvole: nema
		\item Vrijednost: stvarni audio sadržaj
	\end{itemize}
	\item \textit{Att3} sadrži konfiguraciju karakteristika klijenta
	\begin{itemize}
		\item UUID: standardni 16-bitni UUID za karakterističnu konfiguraciju klijenta
		\item Dozvole: R/W
		\item Vrijednost: prvi bit označava omogućenost obavijesti (0 ili 1), drugi bit omogućenost indikacija
	\end{itemize}
\end{itemize}

Usluga \textit{BlueVoiceOPUS} može implementirati odašiljač, prijamnik ili oboje u slučaju \textit{full-duplex} komunikacije. Za ovu aplikaciju potrebno je implementirati odašiljač odnosno transmiter.
Za prijenos zvuka, usluga i karakteristike moraju se kreirati pozivanjem \lstinline|BVOPUS_STM_Init|, što uključuje funkcije \lstinline|BluevoiceOPUS_AddService| i \lstinline|BluevoiceOPUS_AddChar|; UUID-ovi su definirani u datoteci \lstinline|bvopus_service_stm.c|.

Karakteristike se mogu dodati već postojećoj usluzi pozivanjem \lstinline|BluevoiceOPUS_AddChar| i prosljeđivanjem oznake te određene usluge kao parametra. Ako funkcija vrati \lstinline|BV_OPUS_SUCCESS|, BLE profil je ispravno kreiran.

Također, potrebno je konfigurirati Opus koder. U skladu sa traženim funkcijama, koder se može kreirati ispunjavanjem relevantne strukture \lstinline|OPUS_IF_ENC_ConfigTypeDef|.

Koder se može inicijalizirati pozivom \lstinline|BVOPUS_CodecEncInit(&EncConfigOpus)|. Ako funkcija vrati \lstinline|BV_OPUS_SUCCESS|, \textit{BlueVoiceOPUS} profil ispravno je konfiguriran. Ako vrati \lstinline|BV_OPUS_INVALID_PARAM|, neki od parametara nisu ispravni. Ovisno o odabranim parametrima, inicijalizacijska funkcija dodjeljuje količinu memorije koju relevantni API vraća interno. 

Pri inicijalizaciji podržani su sljedeći parametri:
\begin{itemize}
	\item \textit{application}: \lstinline|OPUS_APPLICATION_VOIP, OPUS_APPLICATION_AUDIO, OPUS_APPLICATION_RESTRICTED_LOWDELAY|
	\item \textit{bitrate} [bps]: od 6000 do 510000
	\item \textit{channels}: od 1 do 255
	\item \textit{complexity}: od 0 do 10
	\item \textit{ms\_frame} [ms]: 2.5, 5, 10, 20, 40, 60
	\item \textit{sample\_freq} [Hz]: 8000, 12000, 16000, 24000, 48000
\end{itemize}

\begin{lstlisting}[caption={Parametri za Opus koder}, language=c]
	EncConfigOpus.application = OPUS_APPLICATION_VOIP;
	/* bps */
	EncConfigOpus.bitrate = 24000; 
	/* 1 channel, mono*/
	EncConfigOpus.channels = AUDIO_CHANNELS_IN; 
	EncConfigOpus.complexity = 0;
	/* 20 ms */
	EncConfigOpus.ms_frame = AUDIO_IN_MS; 
	/* 16000 Hz */
	EncConfigOpus.sample_freq = AUDIO_IN_SAMPLING_FREQUENCY; 
\end{lstlisting}

Nakon postavljanja veze, modul koji je otkrio \textit{BlueVoiceOPUS} profil drugog modula mora omogućiti kontrolnu obavijest pozivanjem API-ja \lstinline|BluevoiceOPUS_EnableCtrl_Notif(void)|. Kontrolna se obavijest zatim koristi za zahtjev za pokretanje i zaustavljanje prijenosa.

Za početak audio prijenosa, modul odašiljača mora zatražiti od prijamnika da omogući njegovu audio obavijest pozivom \lstinline|BluevoiceOPUS_SendEnableNotifReq|. Ovaj API šalje obavijest putem kontrolne karakteristike koja sadrži dva bajta (\lstinline|{BV_OPUS_CONTROL, BV_OPUS_ENABLE_NOTIF_REQ}|). Čim čvor primi zahtjev može omogućiti audio obavijest podnositelju zahtjeva pozivom funkcije \lstinline|BluevoiceOPUS_EnableAudio_Notif(void)|. Ako je obavijest ispravno omogućena, modul može započeti prijenos zvuka.

\textit{BlueVoiceOPUS} profil na ulaz prihvaća količinu PCM uzoraka jednaku veličini audio okvira postavljenoj tijekom Opus konfiguracije. Svaki put kada je audio okvir spreman, treba pozvati API \lstinline|BluevoiceOPUS_SendAudioData| i on automatski sažima, fragmentira i šalje pakete audio podataka.

Za svaku primljenu zvučnu obavijest potrebno je pozvati \lstinline|BluevoiceOPUS_ParseData| i provjeriti vraćeni status. U slučaju uspjeha, parametar \lstinline|pcm_samples| pokazuje je li spreman kompletan audio okvir.

Prema zadanim postavkama, Opus koder je konfiguriran s promjenjivom brzinom prijenosa: svaki kodirani okvir ima  duljinu prilagođenu brzini prijenosa postavljenoj tijekom faze inicijalizacije. Maksimalna veličina BLE paketa postavljena je na 150 bajtova, a broj BLE paketa može varirati među različitim audio okvirima ili ovisno o konfiguraciji Opusa.

Protokol prijenosa \textit{BlueVoiceOPUS} modula pokazuje kada kodirani podaci počinju i završavaju tako da prijamnik može ponovno izgraditi komprimirani međuspremnik i dekodirati ga: jedan bajt se dodaje kao prvi bajt svakog BLE paketa, preostalih 19 bajtova ili više, ovisno o odabranom MTU, popunjeni su podacima kodiranim Opusom.

Sljedeće vrijednosti mogu biti bajt zaglavlja:
\begin{itemize}
	\item \lstinline|BV_OPUS_TP_START_PACKET = 0x00|
	\item \lstinline|BV_OPUS_TP_START_END_PACKET = 0x20|
	\item \lstinline|BV_OPUS_TP_MIDDLE_PACKET = 0x40|
	\item \lstinline|BV_OPUS_TP_END_PACKET = 0x80|
\end{itemize}

Protokol prijenosa u potpunosti je obrađen u \textit{BlueVoiceOPUS} usluzi.
\subsection{Opus}
Opus je otvoren i svestran audio kodek koji se može koristiti za različite vrste aplikacija kao što su \textit{streaming} govora i glazbe ili komprimirana pohrana zvuka. Skalabilnost, od uskopojasnog govora niske brzine prijenosa pri 6 kbit/s do stereo glazbe pri 510 kbit/s niske složenosti, čini ga pogodnim za širok raspon interaktivnih aplikacija.

Sastoji se od dva sloja: jedan se temelji na linearnom predviđanju (LP), a drugi se temelji na modificiranoj diskretnoj kosinusnoj transformaciji (MDCT). Opus kombinira rezultate s gubitcima i bez gubitaka. Primjerice, u govornim aplikacijama, LP tehnike poput CELP-a (engl. \textit{Code-excited linear prediction}) učinkovitije kodiraju niske frekvencije nego u tehnikama transformacijske domene kao što je MDCT.

Opus kodek se sastoji od SILK i CELT tehnologija kodiranja. Prvi koristi model temeljen na predviđanju (LPC), dok je drugi u potpunosti modeliran na MDCT transformaciji. Ova svestranost omogućuje Opusu rad u tri načina rada (SILK, CELT ili hibridni način) i osigurava višestruke konfiguracije za različite aplikacije.

\section{Programska potpora za računalo}

Glavna zadaća aplikacije na računalu je primiti audio signal Bluetoothom, prikladno ga obraditi te moći izravno reproducirati i pohraniti. Za računalnu programsku potporu odabrana je \textit{BlueST-SDK} biblioteka koji omogućuje jednostavan pristup podacima koje izvozi BLE uređaj s implementiranim BlueST protokolom. Protokol BlueST lako je proširiv za podršku korisnički definiranih podataka. On već definira različite podatke koji dolaze iz različitih senzora kao što su inercijski senzori, senzori okoliša, informacije o bateriji, te DC i motori. Također implementira serijsku konzolu preko Bluetootha koja omogućuje funkcionalnost \textit{stdint}/\textit{stdout}/\textit{stderr} i definira konfiguracijski servis za kontrolu postavki povezanih ploča. 

Korištenjem zajedničkog modela programiranja za podržane platforme, \textit{BlueST-SDK} olakšava razvoj aplikacija na Android, iOS i Linux (s instaliranim Python) sustavima i uključuje primjere aplikacija koji demonstriraju korištenje SDK. Python izdanje \textit{BlueST-SDK} biblioteke koristi \textit{bluepy} Python biblioteku dostupnu na Linuxu za povezivanje s BLE uređajima.

Za razvoj je odabran programski jezik Python na operacijskom sustavu Linux. 

\begin{figure}[ht]
	\includegraphics[width=\linewidth]{imgs/bluest_stack}
	\caption{Arhitektura aplikacije s \textit{BlueST-SDK} modulom}
	\label{fig:bluest_stack}
\end{figure}
\begin{figure}[ht]
	\includegraphics[width=\linewidth]{imgs/duplex_chain_2}
	\caption{Lanac obrade prijemnika u aplikaciji}
	\label{fig:duplex_chain_2}
\end{figure}
\chapter{Programska potpora za korisničko sučelje}

GUI aplikacija izrađena je korištenjem razvojnog alata PyQt temeljenog na programskom
jeziku Python i pripadnih biblioteka za razvoj grafičkih korisničkih sučelja. PyQt je \textit{plug-in} za Python, binding of the cross-platform GUI toolkit Qt.
\chapter{Obrada audio signala}

Nakon otvaranja audio datoteke započinje proces obrade zvučnog signala. Audio biblioteka \textit{SoundFile} koristi se za učitavanje audio datoteke te, pozivom \lstinline|sf.read(file_path)| dobivaju se matrica amplituda i frekvencija uzorkovanja. Dobivena matrica reprezentacija je audio signala u vremenskoj domeni, odnosno prikazuje glasnoću (amplitudu) zvuka dok se mijenja u vremenu. Amplituda jednaka nuli označava tišinu.

Za analizu odnosa amplitude i frekvencije signala potrebno transformirati signal u frekvencijsku domenu kako bi se prikazalo koje se frekvencije nalaze u signalu. Fourierovom transformacijom signal se dekomponira u odgovarajuće frekvencije. \textit{Scipy} biblioteka sadrži ugrađenu funkciju za brzu Fourierovu transformaciju.

Dobivene matrice iscrtavaju se grafički pomoću biblioteke \textit{matplotlib}. Ograničen je prikaz frekvencija na 2000 Hz...


\begin{figure}[ht]
	\includegraphics[width=\linewidth]{imgs/analyse_example}
	\caption{Primjer prikaza analize audiozapisa}
	\label{fig:analyse_example}
\end{figure}

\section{Odnos intenziteta zvuka i udaljenosti}

\section{Obrada zvučnih zapisa hrkanja}

\chapter{Zaključak}

Na razvojnom sustavu STM32WB5MM-DK implementiran je sustav za prijam i prijenos audio signala. Audio signal snima se MEMS mikrofonom ugrađenim u razvojni sustav te se snimljeni signal prenosi putem BLE sučelja na računalo. Na računalu je razvijena aplikacija koja povezuje razvojni sustav i računalo. Korištenje aplikacije pojednostavljeno je kreiranjem korisničkog grafičkog sučelja. Aplikacija i korisničko sučelje izvode se u operacijskom sustavu Linux. Neovisan rad korisničkog sučelja i dijela aplikacije koje komunicira s razvojnim sustavom STM32WB5MM-DK omogućen je višedretvenim izvođenjem procesa. Korisničko sučelje za interakciju s korisnikom i snimanje zvuka razvijeno je kao \textit{PyQt} aplikacija. Razvojni alat \textit{PyQt} pruža mogućnosti za razvoj korisničkih sučelja namijenjenih za različite platforme, uključujući operacijski sustav Linux. Detaljnom analizom zvučnih zapisa hrkanja ispitana je povezanost frekvencije i mjesta nastajanja hrkanja, kao i ovisnost intenziteta zvuka o udaljenosti izvora zvuka. 

Konfiguracija sustava opisana ovim završnim radom predstavlja alat za snimanje zvuka koji je moguće koristiti u kliničkom okruženju i ostalim ograničenim okruženjima gdje mikrofon i korisnik aplikacije ne mogu biti u istom prostoru. Također, njime je omogućena jednostavna analiza zvučnih zapisa. 

\eject

\bibliography{literatura}{}
\bibliographystyle{fer}

\begin{sazetak}
U ovom radu implementiran je sustav za prijam, prikaz i obradu audio signala korištenjem razvojnog sustava STM32WB5MM-DK. Korištene su biblioteke koje omogućavaju snimanje zvuka MEMS mikrofonom na razvojnom sustavu. Korišteno je BLE sučelje za prijenos audio signala s razvojnog sustava na računalo. Razvijeno je grafičko korisničko sučelje za snimanje zvuka i vizualizaciju ranije snimljenih podataka korištenjem razvojnog alata \textit{PyQt}. Aplikacija se izvodi na operacijskom sustavu Linux. Provedena je analiza zvučnih zapisa hrkanja snimljenih razvijenim sustavom. 

\kljucnerijeci{STM32WB5MM-DK, BLE, MEMS mikrofon, korisničko sučelje, obrada audio signala, hrkanje}
\end{sazetak}

% TODO: Navedite naslov na engleskom jeziku.
\engtitle{Audio Signal Transmission Using BLE Interface of STM32WB5MM-DK Development Kit}
\begin{abstract}
This thesis describes an implementation of a system for receiving, displaying and analysis of audio signal using STM32WB5MM-DK development board. Libraries that enable audio recording with a MEMS microphone on the development board have been used. BLE interface is used for signal transfer from the STM32WB5MM-DK to the computer. A graphical user interface has been developed for audio recording and audio signal visualization, using \textit{PyQt} development tools. The application runs on the Linux operating system. Analysis of snoring audio recorded with the developed system has been performed. 

\keywords{STM32WB5MM-DK, BLE, MEMS microphone, user interface, audio signal processing, snoring}
\end{abstract}

\end{document}
